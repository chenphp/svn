\documentclass[nofonts, oneside]{ctexbook}

\usepackage{xeCJK}
\usepackage{hyperref}
\usepackage{makeidx}

\makeindex
\setCJKmainfont{AR PL UMing CN}
\setCJKsansfont{AR PL UMing CN}
\setCJKmonofont[Scale=0.9]{AR PL UMing CN}
\setmainfont{FreeSerif}
\setsansfont{FreeSans}
% "Mapping={}" make quote symbol straight
\setmonofont[Mapping={}]{FreeMono}

\title{SVN 教程}
\author{\url{www.tutorialspoint.com/svn/index.htm} \and
	\url{https://github.com/wuzhouhui/svn}}

\begin{document}

\maketitle

\chapter{基本概念}
\label{chap:basic_concepts}

\section{什么是版本控制系统}
\label{sec:what_is_version_control_system}

版本控制系统 (Version Control System\index{Version Control System (版本控制
系统)}, 简称 VCS) 是一种软件, 它可以帮助软件开发人员协同工作, 以及管理产品的
完整历史.

版本控制系统的目标包括:
\begin{itemize}
  \item 支持多人同时操作.
  \item 不覆盖其他人作出的修改.
  \item 维护每一个版本的历史.
\end{itemize}

VCS 可以分成两大类别:
\begin{itemize}
  \item 集中式的版本控制系统 (Centralized Version Control System\index{
   Centralized Control System (集中式的版本控制系统)}, 简称 CVCS);
  \item 分布式的版本控制系统 (Distributed Version Control System\index{
   Distributed Version Control System (分布式的版本控制系统)}, 简称 DVCS).
\end{itemize}

本教程只讨论 CVCS, 特别是 Subversion\index{Subversion}, 它使用中央服务器来存
储所有的文件, 并支持团队协作.

\section{版本控制术语}
\label{sec:version_control_terminologies}

首先先来介绍本教程将会用到的几个术语.

\begin{itemize}
  \item 仓库 (Repository\index{repository (仓库)}): 仓库是所有版本控制系统的
      核心, 它是开发人员存放所有资料的中心位置. 除了文件, 仓库还会存放历史.
      仓库支持网络访问, 相当于一个服务器, 而版本控制工具则是客户端. 客户端可
      以连接仓库, 然后就可以向仓库提交修改, 或检索修改历史. 通过提交, 其他客
      户端就可以看到某个客户端作出的修改; 通过检查修改历史, 客户端就可以把其
      他人的修改作为工作副本.

  \item 主干 (Trunk\index{trunk (主干)}): 主干是一个目录, 它是所有主要开发发
      生的地方, 通常会被开发人员检出, 以便进行项目开发.

  \item 标签 (Tags\index{tags (标签)}): 标签是用于存放项目的命名快照的目录. 通
      过标签, 开发人员可以给仓库的某个特定版本取一个描述性的, 易于记忆的名字.

      比如, \texttt{LAST\_STABLE\_CODE\_BEFORE\_EMAIL\_SUPPORT} 就比
      \texttt{Repository UUID: 7ceef8cb-3799-40dd-a067-c216ec2e5247} 和
      \texttt{Revision: 13} 容易记忆.

  \item 分支 (Branches\index{branch (分支)}): 分支用来创建一条新的开发线. 如
      果开发人员想要把开发过程分裂成两个方向, 就会用到该功能. 例如, 开发人员
      在发布了 5.0 版本后, 可能会创建一条新的分支, 专门用于开发 6.0 版本,
      这样的话, 6.0 的开发就不会与 5.0 的问题修复相互混淆.

  \item 工作副本 (Working copy\index{working copy (工作副本)}): 工作副本是仓库
      的一个快照. 仓库被团队内的所有人共享, 但人们不能直接修改仓库, 解决办法是
      每个开发人员都从仓库中检出一份工作副本, 这个工作副本就是他的私有工作区,
      开发人员在工作副本中所做的工作并不会影响到团队中的其他人.

  \item 提交修改 (Commit changes\index{commit changes (提交修改)}): 把私有工作
      区的修改存放到中央服务器的过程称为提交. 提交后, 团队中的其他人就可以看到
      别人作出的修改, 通过检索修改, 开发人员可以把修改更新到他们的工作副本中.
      提交是一个原子操作, 要么全部的修改提交成功, 要么全部失败, 不可能出现只
      提交一半的情况.
\end{itemize}

\printindex

\end{document}
