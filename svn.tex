\documentclass[nofonts, oneside]{ctexbook}

\usepackage{xeCJK}
\usepackage{hyperref}
\usepackage{makeidx}
\usepackage{listings}

\newcommand\shellcmd[1]{\texttt{#1}}

\lstnewenvironment{svnshell}
  {
    \lstset{basicstyle = \ttfamily,
        tabsize = 4,
        xleftmargin = 2em,
        frame = single,
        escapechar = &,
    }
  }
  {}

\makeindex
\setCJKmainfont{AR PL UMing CN}
\setCJKsansfont{AR PL UMing CN}
\setCJKmonofont[Scale=0.9]{AR PL UMing CN}
\setmainfont{FreeSerif}
\setsansfont{FreeSans}
% "Mapping={}" make quote symbol straight
\setmonofont[Mapping={}]{FreeMono}

\title{SVN 教程}
\author{\url{www.tutorialspoint.com/svn/index.htm} \and
	\url{https://github.com/wuzhouhui/svn}}

\begin{document}

\maketitle

\chapter{基本概念}
\label{chap:basic_concepts}

\section{什么是版本控制系统}
\label{sec:what_is_version_control_system}

版本控制系统 (Version Control System\index{Version Control System (版本控制
系统)}, 简称 VCS) 是一种软件, 它可以帮助软件开发人员协同工作, 以及管理产品的
完整历史.

版本控制系统的目标包括:
\begin{itemize}
  \item 支持多人同时操作.
  \item 不覆盖其他人作出的修改.
  \item 维护每一个版本的历史.
\end{itemize}

VCS 可以分成两大类别:
\begin{itemize}
  \item 集中式的版本控制系统 (Centralized Version Control System\index{
   Centralized Control System (集中式的版本控制系统)}, 简称 CVCS);
  \item 分布式的版本控制系统 (Distributed Version Control System\index{
   Distributed Version Control System (分布式的版本控制系统)}, 简称 DVCS).
\end{itemize}

本教程只讨论 CVCS, 特别是 Subversion\index{Subversion}, 它使用中央服务器来存
储所有的文件, 并支持团队协作.

\section{版本控制术语}
\label{sec:version_control_terminologies}

首先先来介绍本教程将会用到的几个术语.

\begin{itemize}
  \item 仓库 (Repository\index{repository (仓库)}): 仓库是所有版本控制系统的
      核心, 它是开发人员存放所有资料的中心位置. 除了文件, 仓库还会存放历史.
      仓库支持网络访问, 相当于一个服务器, 而版本控制工具则是客户端. 客户端可
      以连接仓库, 然后就可以向仓库提交修改, 或检索修改历史. 通过提交, 其他客
      户端就可以看到某个客户端作出的修改; 通过检查修改历史, 客户端就可以把其
      他人的修改作为工作副本.

  \item 主干 (Trunk\index{trunk (主干)}): 主干是一个目录, 它是所有主要开发发
      生的地方, 通常会被开发人员检出, 以便进行项目开发.

  \item 标签 (Tags\index{tags (标签)}): 标签是用于存放项目的命名快照的目录. 通
      过标签, 开发人员可以给仓库的某个特定版本取一个描述性的, 易于记忆的名字.

      比如, \texttt{LAST\_STABLE\_CODE\_BEFORE\_EMAIL\_SUPPORT} 就比
      \texttt{Repository UUID: 7ceef8cb-3799-40dd-a067-c216ec2e5247} 和
      \texttt{Revision: 13} 容易记忆.

  \item 分支 (Branches\index{branch (分支)}): 分支用来创建一条新的开发线. 如
      果开发人员想要把开发过程分裂成两个方向, 就会用到该功能. 例如, 开发人员
      在发布了 5.0 版本后, 可能会创建一条新的分支, 专门用于开发 6.0 版本,
      这样的话, 6.0 的开发就不会与 5.0 的问题修复相互混淆.

  \item 工作副本 (Working copy\index{working copy (工作副本)}): 工作副本是仓库
      的一个快照. 仓库被团队内的所有人共享, 但人们不能直接修改仓库, 解决办法是
      每个开发人员都从仓库中检出一份工作副本, 这个工作副本就是他的私有工作区,
      开发人员在工作副本中所做的工作并不会影响到团队中的其他人.

  \item 提交修改 (Commit changes\index{commit changes (提交修改)}): 把私有工作
      区的修改存放到中央服务器的过程称为提交. 提交后, 团队中的其他人就可以看到
      别人作出的修改, 通过检索修改, 开发人员可以把修改更新到他们的工作副本中.
      提交是一个原子操作, 要么全部的修改提交成功, 要么全部失败, 不可能出现只
      提交一半的情况.
\end{itemize}

\chapter{环境设置}
\label{chap:environment_setup}

\section{安装 SVN}
\label{sec:svn_installation}

Subversion\index{Subversion} 是一款流行的版本控制工具, 它是开源软件, 可以在
因特网上免费获取. 大部分 GNU/Linux 发行版都默认安装了 Subversion, 可以用下面
的命令检查:
\begin{svnshell}
[jerry@CentOS ~]$ svn --version&\index{svn!--version}&
\end{svnshell}
如果系统中已经安装了 Subversion 客户端, 命令就会输出 Subversion 的版本号, 否则
的话, 就会输出一条错误信息:
\begin{svnshell}
[jerry@CentOS ~]$ svn --version&\index{svn!--version}&
-bash: svn: command not found
\end{svnshell}

如果你用的是基于 RPM 的 GNU/Linux 发行版, 可以用命令 \shellcmd{yum}\index{yum}
来安装
Subversion, 安装完成后, 再执行 \shellcmd{svn --version} 检查是否安装成功:
\begin{svnshell}
[jerry@CentOS ~]$ su -
Password:
[jerry@CentOS ~]# yum install subversion&\index{yum!install}&

[jerry@CentOS ~]$ svn --version&\index{svn!--version}&
svn, version 1.6.11 (r934486)
compiled Jun 23 2012, 00:44:03
\end{svnshell}

如果是 Debian 系列的 GNU/Linux 发行版, 就用命令 \shellcmd{apt}\index{apt} 安装:
\begin{svnshell}
    [jerry@Ubuntu]$ sudo apt-get update&\index{apt!update}&
[sudo] password for jerry:

[jerry@Ubuntu]$ sudo apt-get install subversion&\index{apt!install}&

[jerry@Ubuntu]$ svn --version&\index{svn!--version}&
svn, version 1.7.5 (r1336830)
compiled Jun 21 2013, 22:11:49
\end{svnshell}

\section{Apache 设置}
\label{sec:apache_setup}

上面我们已经介绍了如何在 GNU/Linux 中安装 Subversion 客户端, 现在介绍如何创建
一个新的仓库, 并设置访问权限.

在服务器端, 我们需要安装 Apache \shellcmd{httpd}\index{httpd} 模块和
\shellcmd{svnadmin}\index{svnadmin} 工具:
\begin{svnshell}
[jerry@CentOS ~]$ su -
Password:
[root@CentOS ~]# yum install&\index{yum!install}& mod_dav_svn subversion
\end{svnshell}
安装了软件包 \texttt{mod\_dav\_svn}\index{\texttt{mod\_dav\_svn}} 之后, 用户就
可以用 HTTP 来访问仓库, 软件包
\texttt{subversion}\index{subversion} 包含了
\shellcmd{svnadmin}\index{svnadmin} 工具.

\texttt{/etc/httpd/conf.d/subversion.conf}\index{subversion.conf} 是 Subversion
的配置文件, 该文件的典型内容如下:
\begin{svnshell}
LoadModule dav_svn_module   modules/mod_dav_svn.so
LoadModule authz_svn_module modules/mod_authz_svn.so

<Location /svn>
    DAV svn
    SVNParentPath /var/www/svn
    AuthType Basic
    AuthName "Authorization Realm"
    AuthUserFile /etc/svn-users
    Require valid-user
</Location>
\end{svnshell}

现在我们开始创建 Subversion 用户, 并赋予他们访问仓库的权限. 命令
\shellcmd{htpasswd}\index{htpasswd} 用于创建和更新纯文本文件, 这些文件中存放
着用户名和密码. 如果指定了选项 \shellcmd{-c}, 则命令会创建一个密码文件, 如果
密码文件已经存在, 就会被覆盖, 因此选项 \shellcmd{-c} 只在第一次时用到. 选项
\shellcmd{-m} 开启密码的 MD5 加密.

\printindex

\end{document}
