\documentclass[nofonts, oneside]{ctexbook}

\usepackage{xeCJK}
\usepackage{hyperref}
\usepackage{makeidx}

\makeindex
\setCJKmainfont{AR PL UMing CN}
\setCJKsansfont{AR PL UMing CN}
\setCJKmonofont[Scale=0.9]{AR PL UMing CN}
\setmainfont{FreeSerif}
\setsansfont{FreeSans}
% "Mapping={}" make quote symbol straight
\setmonofont[Mapping={}]{FreeMono}

\title{SVN 教程}
\author{\url{www.tutorialspoint.com} \and
	\url{https://github.com/wuzhouhui/hacking_vim} }

\begin{document}

\maketitle

\chapter{基本概念}
\label{chap:basic_concepts}

\section{什么是版本控制系统}
\label{sec:what_is_version_control_system}

版本控制系统 (Version Control System\index{Version Control System (版本控制
系统)} 简称 VCS) 是一种软件, 它可以帮助软件开发人员协同工作, 以及管理产品的
完整历史.

版本控制系统的目标包括:
\begin{itemize}
  \item 支持多人同时操作.
  \item 不覆盖其他人作出的修改.
  \item 维护每一个版本的历史.
\end{itemize}

VCS 可以分成两大类别:
\begin{itemize}
  \item 集中式的版本控制系统 (Centralized Version Control System\index{
   Centralized Control System (集中式的版本控制系统)}, 简称 CVCS);
  \item 分布式的版本控制系统 (Distributed Version Control System\index{
   Distributed Version Control System (分布式的版本控制系统)}, 简称 DVCS).
\end{itemize}

本教程只讨论 CVCS, 特别是 Subversion\index{Subversion}. Subversion 属于 CVCS,
因此它使用中央服务器来存储所有的文件, 并支持团队协作.

\section{版本控制术语}
\label{sec:version_control_terminologies}

首先先来介绍本教程将会用到的几个术语.

\begin{itemize}
  \item 仓库 (Repository\index{repository (仓库)}): 仓库是所有版本控制系统的
      核心, 它是开发人员存放所有资料的中心位置. 除了文件, 仓库还会存放历史.
      仓库支持网络访问, 相当于一个服务器, 而版本控制工具则是客户端. 客户端可
      以连接仓库, 然后就可以向仓库提交修改, 或检索修改历史. 通过提交, 其他客
      户端就可以看到某个客户端作出的修改; 通过检查修改历史, 客户端就可以把其
      他人的修改作为工作副本 (working copy).

  \item 主干 (Trunk\index{trunk (主干)}):
\end{itemize}

\printindex

\end{document}
